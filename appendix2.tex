\section*{Supplementary Materials for Chapter 3: Constrain Paleopoles For A
Certain Tectonic Plate}

A polygon can be drawn around a set of paleomagnetic data, whose sampling sites
we believe belong to a specific plate or rigid block. Then the {\em Spatial
Join\/} technique~\citep{J07} helps join attributes from the polygon to the
paleomagnetic data based on the spatial relationship allowing data within this
polygon to be extracted from the whole raw large dataset without splitting a
subset just for a specific plate. That allows us to quickly select subsets of
the database based on geographic constraints just as easily as for age. Of
course, the boundary of this polygon must be reasonably along a tectonic
boundary. Regions like those close to the plate boundaries are usually
tectonically active (e.g.\ local rotations), so we should also be careful when
we deal with the paleopoles derived from this type of locations.

\subsection{120\textendash0 Ma North America}

The data-constraining polygons are from the recently published plate
model~\citep{Y18} (Fig.~\ref{fig_NAfinal}). Plate ID 101 polygon in the
recently published Plate Model~\citep{Y18}, including its children 108
(Avalon/Acadia block) and 109 (Piedmont block) polygons for 120\textendash0 Ma,
is used to select the sampling sites of the paleopoles for North America.
According to the plate model rotation data~\citep{Y18}, 108 is fixed to 101
during the geologic period from Cretaceous to the present day. 109 is also fixed
to 101 since ${\sim}300$ Ma~\citep{C14}. Then in order to be compared with the
FHM (120\textendash0 Ma)~\citep{M93,M99}, the paleopoles with age ranging
120\textendash0 Ma are further selected through constraining the lower magnetic
age ``LOMAGAGE $<=$ 135'' (here it is not 120 but 135, because for the lower
resolution case when the window length is 30 Myr, the Age Position Picking
method will include those data with their lower magnetic age between 120 Ma and
135 Ma). In addition, the RESULTNO=6007 dataset should also be included
according to a published plate kinematic model~\citep{Mc06}
with a relatively higher resolution of polygons and
rotations, although the dataset is in the PlateID=178 polygon. In the end, 193
datasets in total are extracted (both white circles and red
triangle-inside-circles in Fig.~\ref{fig_NAfinal}).

Also based on this model of southwestern North America since 36 Ma~\citep{Mc06},
part of the paleopoles constrained by the four small western terranes whose
Plate IDs are also 101 (white circles in Fig.~\ref{fig_NAfinal}) in fact had gone through regional
rotations and here are removed. However, the poles with age younger than 10 Ma
located within the largest western 101 terrane (on the south of the smallest
western 101 terrane; corresponding to the RANGE\_ID=74 polygon in the
model~\citep{Mc06}) should be included. So finally 135 of the 193 datasets remain
(Fig.~\ref{fig_NAfinal}). Spatially North American paleomagnetic data are mainly
from the western and eastern margins of the plate.

\begin{figure}
\includegraphics[scale=.4]{../../paper/tex/GeophysJInt/figures/Appendix/101.pdf}
\caption{The final filtered datasets (red triangle-inside-circles) for later
analysis on 120\textendash0 Ma North America. Those poles that had been
influenced by local tectonic rotations are shown as white
circles.}\label{fig_NAfinal}
\end{figure}

\subsection{120\textendash0 Ma India}

Plate ID 501 polygons in the recently published Plate Model~\citep{Y18} also
include the two small polygons of the northern ``Lesser Himalayan passive
margin of Greater Indian Basin'' and ``Tethyan Himalayan microcontinent of
Greater India'' (Fig.~\ref{fig_INfinal}). The polygons are used to select the
sampling sites of the paleopoles for India (Fig.~\ref{fig_INfinal}).

\begin{figure}
\includegraphics[scale=.4]{../../paper/tex/GeophysJInt/figures/Appendix/501.pdf}
\caption{The final filtered datasets (red triangle-inside-circles) for later
analysis on 120\textendash0 Ma India. Those poles that had been influenced by
local tectonic rotations are shown as white circles. The rifts, faults and
detachments (red lines) around India are used to filter out those data that
are influenced by local tectonic rotations.}\label{fig_INfinal}
\end{figure}

Based on the model of the tectonic interactions between India, Arabia and Asia
since the Jurassic~\citep{G15} (Fig.~\ref{fig_INfinal}), part of the paleopoles
constrained by the north two small terranes whose Plate IDs are also 501 in fact
had gone through regional rotations and here are removed. So finally 75 datasets
are left (Fig.~\ref{fig_INfinal}). Spatially Indian paleomagnetic data are more
evenly distributed on the India plate than North American and Australian poles.

\subsection{120\textendash0 Ma Australia}

Plate ID 801 polygon in the recently published Plate Model~\citep{Y18}, including
its children 675 (Sumba block) and 684 (Timor block) polygons for
120\textendash0 Ma (Fig.~\ref{fig_AUfinal}), is used to select the sampling
sites of the paleopoles for Australia. According to the plate model rotation
data~\citep{Y18}, 675 and 684 are fixed to 801 during the geologic period from
${\sim}145$ Ma to the present.

On the southeast of the main Australia plate (the blue polygon in
Fig.~\ref{fig_AUfinal}), there is a triangle-shaped small polygon 850
(Tasmania block) which is fixed to 801 since ${\sim}100$ Ma according to
the~\citet{Y18} rotation data. With that attribute, 805 contributes more data
younger than ${\sim}100$ Ma for the later analysis. Ultimately the final 99 extracted
datasets is shown in Fig.~\ref{fig_AUfinal}.

\begin{figure}
\includegraphics[scale=.4]{../../paper/tex/GeophysJInt/figures/Appendix/801.pdf}
\caption{The final filtered datasets (red triangle-inside-circles) for later
analysis on 120\textendash0 Ma Australia. Those poles that had been influenced
by local tectonic rotations are shown as white circles. The Plate ID 850 helps
increase the amount of qualified datasets for 100\textendash0
Ma.}\label{fig_AUfinal}
\end{figure}

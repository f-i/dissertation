\chapter{Methodologies}\label{chap:Metho}
\textit{This chapter mainly describes the development of a new APW path
similarity measuring tool used throughout the dissertation.
Apparent polar wander paths (APWPs) based on paleomagnetic data are the
principal means of describing plate motions through most of Earth history.
Comparing the spatio-temporal patterns and trends of APWPs between different
tectonic plates is important for testing proposed paleogeographic
reconstructions of past supercontinents. However, thus far there is no clearly
defined quantitative approach to determine the degree of similarity between
APWPs. This paper proposes a new method of determining the degree of similarity
between two APWPs that combines three separate difference metrics that assess
both spatial separation of coeval poles, and similarities in the bearing and
length of coeval segments using a weighted linear summation. Bootstrap tests are
used to determine whether the differences between coeval poles and segments are
significant for the given spatial uncertainties in pole positions. The Fit
Quality is used to discriminate between low significance scores caused by
comparing poorly constrained paths with large spatial uncertainties from those
caused by a close fit between well-constrained paths. The individual and
combined metrics are demonstrated using tests on synthetic pairs of APWPs with
varying degrees of spatial and geometric difference. In a test on real
paleomagnetic data, we show that these metrics can quantify the effects of
correction for inclination shallowing in sedimentary rocks on Gondwana and
Laurussia's 320\textendash0 Ma APWPs. For an APWP pair, when one APWP's three
individual metrics are all greater than or equal to, or less than or equal to
the other one's, weighting is dispensable because the similarity ranking order
becomes straightforward; otherwise assigning equal weights is recommended,
although then decision makers are allowed to arbitrarily change weights
according to their preferences.}
\vfill
\minitoc\newpage

(This chapter is also openly accessible from
\emph{https://github.com/f-i/APWP\_similarity}. Text:
\emph{https://github.com/f-i/APWP\_similarity/blob/master/2.pdf}; Figures:
\emph{https://github.com/f-i/APWP\_similarity/blob/master/2\_figures.pdf};
Supplementary:
\emph{https://github.com/f-i/APWP\_similarity/blob/master/2Supp.pdf})


\section{Introduction}

Paleomagnetism is an important source of information on the past motions of the
Earth's tectonic plates. The orientation of remanent magnetisations acquired by
rocks during their formation record the past position of the Earth's magnetic
poles. In older rocks, these virtual geomagnetic poles often appear to be
increasingly offset from the modern day geographic poles. Because the Earth's
geomagnetic field appears to have remained largely dipolar and centered on the
spin axis for at least the last 2 billion years~\cite{E06}, this divergence is
interpreted as recording the translation and rotation of a continent by the
motion of tectonic plates in the time since the rock formed. An Apparent Polar
Wander Path (APWP) is a time sequence of paleomagnetic poles (or, more commonly,
mean poles that average all regional paleopoles of similar age) that traces
the cumulative motion of a continental fragment relative to the Earth's spin
axis.

Investigations of the Earth's past tectonic evolution and paleogeography often
involve comparing APWPs. For example, if two now separated continental fragments
were once part of the same supercontinent, their APWPs should share the same
geometry during the interval that this supercontinent existed. If the
supercontinent has been correctly reconstructed, the APWPs should also overlap
during this interval (Fig.~\ref{fig:nambal_same_geometry}). APWP comparisons can
be used to assess plate motion models generated using different datasets and/or
fitting techniques~\cite[for example]{B02,B07,S07,T08,D11}; significant deviations
from the known APWP for a continent can also be used to identify local tectonic
rotations~\cite[for example]{G10,Ch13}. Despite the clear importance of measuring
APWP similarity, these comparisons remain largely qualitative in nature,
involving visual comparisons of specific APWP segments and checking if they have
overlapping 95\% confidence limits~\cite[for example]{B02,B07,G10,D11}. Where
quantitative measures are used, the mean great circle distance (GCD) between
coeval poles on the APWP pair has been commonly used as a generalised difference
metric for a pair of APWPs, with a lower score indicating that they are more
similar~\cite[for example]{S07,T08}. However, because GCD is simply a measure of
spatial separation and does not incorporate geometric information about the two
paths being compared, it is possible for pairs with clearly different
similarities to have similar mean GCD scores
(Fig.~\ref{fig:QualitativelyDifferentSameGCD}). Due to the inherent time
variability of the geomagnetic field, uncertainties arising from the sampling
and measurement of remanent magnetisations, and uncertainties in the
magnetization age, the mean paleopoles that make up an APWP also have associated
spatial uncertainties. The significance of a GCD score is therefore not
immediately obvious. A score that indicates a relatively large difference
between two paths may not be significant if the spatial uncertainties are large;
a small difference could be significant if the spatial uncertainties are small
(Fig.~\ref{fig:GCDlargeButIndistinguishable}).


\begin{figure*}[tbp]
\centering
\includegraphics[width=1.01\textwidth]{../../paper/tex/ComputGeosci/figures/nambal_same_geometry.pdf}
\caption[Parts of APWPs of supercontinent fragments share the same
geometry]{(a) The APWPs for North America (black) and Baltica (grey) are
spatially distinct, but their Late Paleozoic\textendash{}Early Mesozoic sections
are geometrically similar due to them both being part of the supercontinent
Pangaea. (b) Reversing the opening of the Atlantic Ocean by rotation around a
reconstruction pole (blue star) results in the overlap of these two APWPs
between 390 million years ago (Ma) and 220 Ma, validating the proposed
paleogeography. The effects of this rotation on Baltica and its APWP (BAL) are
illustrated by the motion of the circle marker (before: blank center; after:
dark center), respectively. General Perspective projection. APWPs and rotation
parameters from~\cite{To16}.}\label{fig:nambal_same_geometry}
\end{figure*}

\begin{figure*}[tbp]
  \captionsetup[subfigure]{labelformat=empty,aboveskip=-6pt,belowskip=-6pt}
  \centering
  \begin{subfigure}[htbp]{1.01\textwidth}
    \centering
    \includegraphics[width=1.01\linewidth]{../../paper/tex/ComputGeosci/figures/QualitativelyDifferentSameGCD.pdf}
    \caption{}\label{fig:QualitativelyDifferentSameGCD}
  \end{subfigure}
  \begin{subfigure}[htbp]{1.01\textwidth}
    \centering
    \includegraphics[width=1.01\linewidth]{../../paper/tex/ComputGeosci/figures/GCDlargeButIndistinguishable.pdf}
    \caption{}\label{fig:GCDlargeButIndistinguishable}
  \end{subfigure}
  \caption[Examples showing GCD is a bad index of similarity]{(a) How the
  average GCD similarity metric ignores path geometry: \emph{Pair}\textbf{1}
  (circles and squares, left) is clearly more similar than \emph{Pair}\textbf{2}
  (circles and triangles, right), but for both pairs each GCD remains constant.
  (b) How GCD also ignores spatial uncertainties. The average GCD separation
  between coeval points is smaller for \emph{Pair}\textbf{1} (circles and
  squares, left) than \emph{Pair}\textbf{2} (circles and triangles, right). But
  if spatial uncertainties (plotted as 95\% confidence ellipses) are considered,
  this ranking is not trustworthy: it is \emph{Pair}\textbf{2} that is
  statistically indistinguishable from the reference path. Azimuthal
  Orthographic projection.}\phantomsection\label{fig:GCDbadIndex}
\end{figure*}

We have developed an improved quantitative method of calculating the similarity
between two APWPs, or coeval segments of APWPs, in the form of a composite
difference score that compares both their spatial overlap and geometry. Our
method incorporates statistical significance testing, allowing paths with
associated spatial uncertainties to be rigorously compared to each other. The
validity and effectiveness of this method, and its superior discrimination
compared to a mean GCD score, are demonstrated by comparing the published APWP
of the North America Plate to seven derivative paths with different degrees of
spatial and geometric noise applied (Fig.~\ref{fig:2traj}).

We also test our algorithm on real paleomagnetic data, demonstrating that this
tool can be used to quantitatively assess the effects of different corrections
(in this case, bulk corrections for inclination shallowing in sediments) on the
similarity between APWPs from different continents.

\section{Methods}

\subsection{Comparing Apparent Polar Wander Paths}

An APWP consists of a sequence of ($\mathbf{n}$) mean poles,
$\mathbf{P_1,P_2,\ldots,P_n}$, which average the published paleopoles from a
particular continent for a particular time interval. Each mean pole has
associated longitude ($\phi$), latitude ($\lambda$), and age ($\mathbf{t}$).
Spatial uncertainty is represented by a 95\% confidence ellipse described by
semi-major axis $\mathbf{dm}$ with azimuth $\beta$ (angle east of north) and
perpendicular semi-minor axis $\mathbf{dp}$ (e.g. Fig.~\ref{fig:2traj}).

\begin{figure*}[tbp]
  \captionsetup[subfigure]{singlelinecheck=off,justification=raggedright,aboveskip=-6pt,belowskip=-6pt}
  \centering
  \begin{subfigure}[htbp]{.495\textwidth}
    \centering
    \caption{}\label{fig:2traj1}
    \includegraphics[width=1\linewidth]{../../paper/tex/ComputGeosci/figures/pairA.pdf}
  \end{subfigure}
  \vspace{.5em}
  \begin{subfigure}[htbp]{.495\textwidth}
    \centering
    \caption{}\label{fig:2traj2}
    \includegraphics[width=1\linewidth]{../../paper/tex/ComputGeosci/figures/pairB.pdf}
  \end{subfigure}
  \begin{subfigure}[htbp]{.495\textwidth}
    \centering
    \caption{}\label{fig:2traj3}
    \includegraphics[width=1\linewidth]{../../paper/tex/ComputGeosci/figures/pairC.pdf}
  \end{subfigure}
  \vspace{.5em}
  \begin{subfigure}[htbp]{.495\textwidth}
    \centering
    \caption{}\label{fig:2traj4}
    \includegraphics[width=1\linewidth]{../../paper/tex/ComputGeosci/figures/pairD.pdf}
  \end{subfigure}
  \begin{subfigure}[htbp]{.495\textwidth}
    \centering
    \caption{}\label{fig:2traj5}
    \includegraphics[width=1\linewidth]{../../paper/tex/ComputGeosci/figures/pairH.pdf}
  \end{subfigure}
  \vspace{.5em}
  \begin{subfigure}[htbp]{.495\textwidth}
    \centering
    \caption{}\label{fig:2traj6}
    \includegraphics[width=1\linewidth]{../../paper/tex/ComputGeosci/figures/pairE.pdf}
  \end{subfigure}
\end{figure*}%
\begin{figure*}[tbp]\ContinuedFloat
  \captionsetup[subfigure]{singlelinecheck=off,justification=raggedright,aboveskip=-6pt,belowskip=-6pt}
  \centering
  \begin{subfigure}[htbp]{.495\textwidth}
    \centering
    \caption{}\label{fig:2traj7}
    \includegraphics[width=1\linewidth]{../../paper/tex/ComputGeosci/figures/pairF.pdf}
  \end{subfigure}
  \caption[Eight examples of APWP comparisons]{APWP pairs used to validate new
path comparison method. In each case the Phanerozoic APWP for Laurentia/North
America (squares, bold line) at 10 million-year (Myr) timesteps~\cite[``RM''
column of its Table 3]{T12}, is compared to a transformed copy (circles, dashed
line): (a) 1\degree\ finite rotation applied to all the mean poles and their 95\%
uncertainty ellipses around an Euler pole at (125\degree E, 88.5\degree S); (b)
as (a), but 15\degree\ rotation around same Euler pole; (c) after rotation as in
(b), the orientation of each APWP segment is randomised whilst keeping their GCD
length and the coeval poles' GCD fixed; (d) after rotation as in (b), the
bearing between coeval poles is randommised whilst keeping their GCD spacing
fixed; (e) as (c), but with randomly varied and relatively smaller associated
spatial uncertainty; (f) as (a), but 45\degree\ rotation around same Euler pole;
(g) as (f), but with zero associated spatial uncertainty. Azimuthal Orthographic
projection.}\phantomsection\label{fig:2traj}
\end{figure*}

For two continents that were once part of a supercontinent, their APWPs for the
period should perfectly overlap when rotated into a common reference frame
(Fig.~\ref{fig:nambal_same_geometry}). However, due to (i) the spatial
uncertainty associated with the mean poles that form an APWP, (ii) differences
in the density and quality of data available to calculate paleopoles for
different continents in coeval time periods, and (iii) uncertainties and
possible errors in the rotations used to represent past plate motions, a perfect
match will not be obtained. Instead, two `matching' paths should share a
generally similar geometry, and largely overlap with each other when rotated
into a common reference frame. A quantitative measure of the spatial and
geometric similarity between these two paths should ideally allow us to
distinguish between non-identical paths that are similar within the associated
uncertainties, and non-identical paths that are actually different, due to
differential motion between the two continents or a poorly constrained
reconstruction.

\subsection{APWP Pairs Used in This Study}

To assess the performance of the evaluation method developed here, we apply it
to seven different path scenarios (Fig.~\ref{fig:2traj}) generated from
transformations of the 530\textendash0 Ma Phanerozoic APWP for
Laurentia~\cite{T12}. Almost exactly identical paths generated by rotating one
by a degree around an Euler pole at (-55\degree, 88.5\degree)
(\emph{Pair}~\textbf{\subref{fig:2traj1}}, Fig.~\ref{fig:2traj1}) represent an
ideal case of matching paths in the same spatial reference frame. Matching paths
that have been rotated out of the same reference frame by small (15\degree;
\emph{Pair}~\textbf{\subref{fig:2traj2}}, Fig.~\ref{fig:2traj2}) and large
(45\degree, \emph{Pair}~\textbf{\subref{fig:2traj6}}, Fig.~\ref{fig:2traj6})
amounts around the same rotation pole represent small and large reconstruction
errors, respectively. Random noise added to the path
(\emph{Pair}~\textbf{\subref{fig:2traj3}}, Fig.~\ref{fig:2traj3},
\emph{Pair}~\textbf{\subref{fig:2traj4}}, Fig.~\ref{fig:2traj4}) or the
associated uncertainties (\emph{Pair}~\textbf{\subref{fig:2traj5}},
Fig.~\ref{fig:2traj5}) represent differences in data source and/or quality. In
the final pair (\emph{Pair}~\textbf{\subref{fig:2traj7}},
Fig.~\ref{fig:2traj7}), spatial noise has been reduced by removing
\emph{Pair}~\textbf{\subref{fig:2traj6}}'s pole uncertainties.

These seven cases allow evaluation of the performance of any path comparison
metric across a range of different spatial and geometric similarities. To be
successful, such a metric must distinguish pairs with high spatial and geometric
similarity (\emph{Pair}~\textbf{\subref{fig:2traj1}}) from pairs with lower
spatial (\emph{Pair}~\textbf{\subref{fig:2traj2}},
\emph{Pair}~\textbf{\subref{fig:2traj6}},
\emph{Pair}~\textbf{\subref{fig:2traj7}})
or geometric (\emph{Pair}~\textbf{\subref{fig:2traj3}},
\emph{Pair}~\textbf{\subref{fig:2traj4}}) or both
(\emph{Pair}~\textbf{\subref{fig:2traj5}}) similarities.

To achieve more robust discrimination than the mean GCD, we propose combining a
metric for spatial misfit (Mean Significant Spatial Difference) with metrics for
geometric difference (Mean Significant Length and Angular Differences) using a
weighted linear summation, as described in the following sections.

\subsection{Significant Spatial Difference}\label{sec:sigDs}

As in previous quantitative comparisons~\cite[for example]{S07,T08}, the spatial
separation of two APWPs is defined by the average GCD distance between their
coeval poles, but we add a filter for spatial uncertainty based on the bootstrap
approach~\cite{T91}. 1000 bootstrapped mean directions for each pole in a
coeval pair were generated (the exact sampling method is dependent on the
available information for the pole\textemdash{}see Supplementary Information for
a full description) and their cumulative distributions in Cartesian coordinates
were compared~\cite{T91}. Pairs that could not be distinguished at the
95\% confidence interval had their GCD separation set to 0 prior to calculation
of the mean GCD distance for all pairs. This distance is then normalised by
dividing by 50\degree, which is referred to the empirical fact that a 95\%
confidence ellipse major semi-axis of about 25\degree\ is considered unacceptably
large by paleomagnetists~\cite{B92}, to obtain the significant spatial
difference $d_s$. A $d_s$ of zero indicates that the two paths are statistically
indistinguishable from each other.

\subsection{Shape Difference}

The shape of an APWP is determined by the orientations and lengths of its
geodesic segments, which are related to the location of the Euler stage pole
that describes plate motions, and the rotation rate about that pole,
respectively. The geometric similarity of two APWPs can therefore be assessed
by comparing (i) the bearings, and (ii) the lengths of their coeval segments
(Fig.~\ref{fig:direcdiff}), with the assumption that similar geometries are
generated by a common set of stage rotations.

\begin{figure}[tbp]
\includegraphics[width=1\linewidth]{../../paper/tex/ComputGeosci/figures/dg1.pdf}
\caption[Geometric difference definition between two APWPs]{Geometric
differences between coeval sections of two different APWPs ({\bf
seg$_1^I$-seg$_2^I$-seg$_3^I$} \& {\bf seg$_1^{II}$-seg$_2^{II}$-seg$_3^{II}$})
can be described by comparing segment lengths (e.g.\ {\bf l$_2^I$} vs. {\bf
l$_2^{II}$}) or changes in bearing of coeval segments relative to their
previous segment (e.g.\ {\bf $\alpha_3^I$} vs. {\bf $\alpha_3^{II}$}). Segments
are along great circles (blue dashed lines). Azimuthal Orthographic
projection.}\label{fig:direcdiff}
\end{figure}

\subsubsection{Mean Length Difference}
The mean length difference between the two APWPs $traj^I$ and $traj^{II}$ is the
absolute sum of differences between the lengths of coeval path segments (e.g.
$l_2^I$ vs $l_2^{II}$, $l_3^I$ vs $l_3^{II}$, $l_4^I$ vs $l_4^{II}$,
Fig.~\ref{fig:direcdiff}), normalised by dividing by the possible maximum
distance the pole could wander during the whole period, such that:
%
\begin{equation*}
  d_l = \frac{\sum\limits_{k=2}^n | l_k^I - l_k^{II} |}{D_{polar} * (t_n-t_1)} ,
  \quad\forall k \in \{2,3,\ldots,n\},
\label{eq:ld}
\end{equation*}
%
where $|l_k^I - l_k^{II}|$ is the length difference of one pair of coeval
segments for an APWP pair ($traj^I$ and $traj^{II}$), e.g. $|l_2^I - l_2^{II}|$
for the beginning coeval segment pair. The normalising parameter $D_{polar}$ is
2.7\degree/Myr, derived from estimates of magnitude of maximum plate
velocity~\cite[up to about 30 cm/year]{S09,K14}. A $d_l$ approaching 1 would
result from a comparison between a virtually stationary APWP and one associated
with a rapidly moving plate.

\subsubsection{Mean Angular Difference}

The mean angular difference describes the degree of consistency between the
polar-wandering directions (defined as the bearing of the older pole in a
segment with respect to the younger one) of two APWPs. In order to robustly
compare two APWPs that have not necessarily been rotated into the same reference
frame, it is more useful to define the APWP geometry relative to the path
itself, rather than an external reference frame. Therefore the bearing of a
segment is expressed as the change in geographic bearing with respect to the
previous segment ($\alpha_3$ and $\alpha_4$, Fig.~\ref{fig:direcdiff}). For
example, $\alpha_3^I$ is the result of subtracting the geographic azimuth
$\theta_{2y}^I$ from $\theta_{1o}^I$, where ``y'' stands for young end of
segment and ``o'' for old end of segment. The first segment cannot record a
relative bearing change: a path with n poles therefore consists of n-1 segments
which are described by n-2 relative angles. The defined range of bearing values
is set as -180\degree\ to 180\degree, with clockwise (east) changes in direction
defined as positive, e.g. {\bf $\alpha_3^{II}$} and {\bf $\alpha_4^{II}$}, and
anticlockwise (west) changes defined as negative, e.g. {\bf $\alpha_3^I$} and
{\bf $\alpha_4^I$}.

The mean angular difference $d_a$ between two paths $traj^I$ and $traj^{II}$ can
then be defined as
%
\begin{equation*}
  d_a = \frac{\sum\limits_{k=3}^n \Delta\alpha_k}{180* (n-2)},
\label{eq:ad}
\end{equation*}
%
where
%
\begin{equation*}
\Delta\alpha_k =
\left\{
\begin{array}{@{}ll@{}}
| \alpha_k^I - \alpha_k^{II} |, & \text{if}\ | \alpha_k^I -
\alpha_k^{II} | \leq180; \\
360 - | \alpha_k^I - \alpha_k^{II} |, &
  \text{otherwise}.\quad\quad\quad\quad\quad\forall k \in \{3,4,\ldots,n\}.
\end{array}
\right.\label{eq:diffAziChange}
\end{equation*}

$d_a$ is normalised by the maximum possible angular deviation of 180\degree. A
score of 0 indicates exactly matching changes in the bearing of coeval segments
along the length of the two paths, and a score of 1 indicates all segment
bearings are antiparallel.

\subsubsection{Significance Testing of Shape Difference}\label{sec:shapeSigTest}

Due to associated spatial uncertainty, the mean poles in an APWP trace out one
possible path within a range of possible geometries (Fig.~\ref{fig:Fig5a}). If
the length and angular difference scores for one path fall within the range of
possible scores for the other, two APWPs may not in fact be significantly
different from each other. Significance testing for the shape difference scores
is performed on each coeval segment pair as follows (Fig.~\ref{fig:Fig5b}):

\begin{figure*}[tbp]
  \captionsetup[subfigure]{singlelinecheck=off,justification=raggedright,aboveskip=-6pt,belowskip=-6pt}
  \centering
  \begin{subfigure}[htbp]{.495\textwidth}
    \centering
    \caption{}\label{fig:Fig5a}
    \includegraphics[width=1\linewidth]{../../paper/tex/ComputGeosci/figures/Fig5a.pdf}
  \end{subfigure}
  \begin{subfigure}[htbp]{.495\textwidth}
    \centering
    \caption{}\label{fig:Fig5b}
    \includegraphics[width=1\linewidth]{../../paper/tex/ComputGeosci/figures/Fig5b.pdf}
  \end{subfigure}
\caption[Testing on Geometry]{Significance testing for the geometric metrics,
$d_l$ and $d_a$. (a) Illustration of how paths traj$^I$ and traj$^{II}$ can be
re-sampled within their uncertainty ellipses, with B being a possible
trajectory of traj$^I$ and C being a possible trajectory of traj$^{II}$. (b)
Upper: If the 95\% confidence interval (black vertical lines are its upper and
lower bounds) for the distribution of difference scores HIST1, generated by
comparing multiple resamplings of traj$^I$ with the original trajectory (A vs B)
does not overlap with the 95\% confidence interval (bounded by white vertical
lines) for the distribution of scores HIST2, generated by comparing resamplings
of traj$^I$ and traj$^{II}$ (B vs C), then the original difference score for
traj$^I$ and traj$^{II}$ is statistically distinguishable; Lower: If the
confidence intervals overlap, then the two paths are not
distinguishable.}\label{fig:Fig5}
\end{figure*}

\begin{itemize}
\item A bootstrapped distribution of possible geometries for each segment in a
  path can be created by resampling the two mean poles that define the original
  segment, in the same manner as described in Section~\ref{sec:sigDs} and the
  Supplementary Information.
\item A histogram of statistically indistinguishable length and/or angular
  difference scores (HIST1, Fig.~\ref{fig:Fig5b}) is created by comparing the
  resampled paths with the original for each $traj^I$ segment.
\item This distribution is then compared to the histogram of difference scores
  created by resampling the coeval segments of $traj^I$ and $traj^{II}$ (HIST2,
  Fig.~\ref{fig:Fig5b}).
\item If the two bootstrapped distributions HIST2 and HIST1 do not overlap at
  the given significance level (e.g.\ the upper and lower bounds of a 95\%
  confidence intervals, Fig.~\ref{fig:Fig5b}), then the difference score is
  interpreted to be significant. If not, then the bearings or lengths of the
  coeval segments are indistinguishable.
\end{itemize}

These tests allow a filter for spatial uncertainty to be added to the $d_a$ and
$d_l$ metrics: prior to summation and normalisation, the difference score is set
to zero for the coeval segments of $traj^I$ and $traj^{II}$ that are
statistically indistinguishable.

\subsection{Composite Path Difference}

The three difference measures described above can be combined into a composite
path difference ($\mathcal{CPD}$) by means of a simple linear weighting rule,
%
\begin{equation*}
\mathcal{CPD} = W_s \cdot d_s + W_a \cdot d_a + W_l \cdot d_l
\label{eq:cpd}
\end{equation*}
%
for $0 < W_s,W_a,W_l < 1$, where $W_s$, $W_a$ and $W_l$ are weighting
coefficients that sum to 1. Different weighting values allow the relative
influences of spatial and geometric (length and angular) similarity to be
varied (Section~\ref{sec:wDis}).

\subsection{Fit Quality}

The three metrics are all tested to be significant based on the spatial
uncertainties. However, the larger the uncertainties are, the less trustworthy
the significant difference scores are (Fig.~\ref{fig:FitQ}). Accordingly, we
bring in a concept of ``Fit Quality'', along the classification scheme of the
reversal test~\cite{M90}. For each mean pole of an APWP, it assigns a score
based on the size of the spatial uncertainty (radius: A95, or (DM+DP)/2): 1 if
it is $\leq$5\degree, 2 if 5\degree$<$r$\leq$10\degree, 3 if
10\degree$<$r$\leq$20\degree, and 4 if it is $>$20\degree. These values are
averaged for each APWP to give a ``Fit Quality'' score (from 1 to 4) for the
difference score. This is then converted into an A/B/C/D letter grade, A if the
average is $<$1.5, B if 1.5$\leq$avg$<$2.5, C if 2.5$\leq$avg$<$3.5, and D if it
is $\geq$3.5, to indicate how easy it is to generate a low difference score. In
other words, an A grade indicates that most poles are well-constrained and so it
is fairly hard to have an indistinguishable path and a low difference score. A D
grade indicates that most poles have large uncertainties so it is much easier to
have a low difference score.

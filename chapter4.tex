\chapter{How Much Data Needed to Make a Reliable APWP}\label{chap:DatNeed}
\textit{This chapter mainly describes how the mean poles with their original
paleopoles at random reduced densities can make a reliable APWP\@. Further we
will see how much data (raw paleopoles) are needed on earth to make a reliable
APWP, and how the ``bad'' paleopoles influence the final result when we have
less data. Are we be able to make a final determination of best number of
paleopoles in each sliding window in average for moving-averaging out an APWP?
(No, different situations for different continents.)}
\vfill
\minitoc\newpage

In the past, especially in deep time, the density and quality of paleomagnetic
data are lower, compared with younger geological times. Reducing the data
density can help see if our methodology is still able to reliably give
reasonable results from data aged in deep times.

\section{Reference Path}

The fixed hotspot model and related plate circuit predicted APWPs are used as
references. In fact, as mentioned in the last chapter, choosing FHM or MHM does
not make much difference at all.

\section{Extration Fraction}

Sub-sampling (extracting part of raw paleopoles) is implemented before
moving-averaging with filtering/correcting and weighting at four percentages,
80\%, 60\%, 40\% and 20\%, which mean 20, 40, 60 and 80 per cent of raw
paleopoles are removed. This means not all sub-samples at, for example, 80\%
are going to be generating a path from the same number of paleopoles after
filtering. In some cases a large number might be removed, in others much less,
depending on the properties of the sub-sampled population. This is definitely an
additional factor that would affect the difference score.

\begin{figure}
    \centering
        \includegraphics[width=0.88\textwidth]{fig/Fay18_10_5.pdf}
    \captionsetup{width=.95\textwidth}
    \caption{Random paleopole samplings (30 times) for the best and worst
	results for the 10 Myr window and 5 Myr step paleomagnetic APWPs vs FHM \&
	plate circuit predicted APWP\@. The lower and upper bound lines connect the
	1st and 3th quantiles ($Q_1$ and $Q_3$) of the 30 samples. The bold line
	connects their means. The numbers in small parentheses are actual quantity
	of paleopoles after filtered by the corresponding picking methods for the
	case with no data removal. The $Q_1$\textendash$Q_3$ interquartile range
	from best method is also shown (shadowed) in the plot of the worst method
	for clarity. Black rings are the lowest value for each method or the lowest
	for the 20\% case.}\label{Fig:Fay18_10_5bw}
\end{figure}

\begin{figure}
    \centering
        \includegraphics[width=0.88\textwidth]{fig/Fay18_10_5to101bw.pdf}
    \captionsetup{width=.95\textwidth}
    \caption{Comparisons of results from the best and worst methods for North
	America (101), also applied on the other two continents (501 and 801). The
	$Q_1$\textendash$Q_3$ interquartile range from Picking No. 11 is also shown
	(shadowed) in the plot of Picking No. 16 for clarity.}\label{Fig:Fay18_10_5to101bw}
\end{figure}

\begin{figure}
    \centering
        \includegraphics[width=0.88\textwidth]{fig/Fay18_10_5to501bw.pdf}
    \captionsetup{width=.95\textwidth}
	\caption{Comparisons of results from the best and worst methods for India
	(501), also applied on the other two continents (101 and 801). The
	$Q_1$\textendash$Q_3$ interquartile range from Picking No. 19 is also shown
	(shadowed) in the plot of Picking No. 8 for clarity.}\label{Fig:Fay18_10_5to501bw}
\end{figure}

\begin{figure}
    \centering
        \includegraphics[width=.88\textwidth]{fig/Fay18_10_5to801bw.pdf}
    \captionsetup{width=.95\textwidth}
    \caption{Comparisons of results from the best and worst methods for
	Australia (801), also applied on the other two continents (101 and 501). The
	$Q_1$\textendash$Q_3$ interquartile range from Picking No. 17 is also shown
	(shadowed) in the plot of Picking No. 22 for clarity.}\label{Fig:Fay18_10_5to801bw}
\end{figure}

We can see that the best picking and weighting methods are statistically always
better than the worst ones even if only 20 percent of thei 120\textendash0 Ma
paleopoles are used to compose the APWPs (Fig.~\ref{Fig:Fay18_10_5bw}) for the
three continents, North America, India and Australia.

For the worst methods applied onto Indian and Australian data, the equal-weight
$\mathcal{CPD}$ surprisingly decreases when the percentage of extracted data
decreases (Fig.~\ref{Fig:Fay18_10_5bw}). This is because after the data density
is reduced the left data are not always enough to cover all the time range of
120\textendash0 Ma but only part, or even though the 120 and 0 Ma mean poles
(two ends) exist, the number of intermediate mean poles between 120 Ma and 0 Ma
is much less than the APWP from data without reducing density.

\subsection{Number of Samples}
Here because the thousands of times of testing for each percentage of data
removal and each picking and weighting method is rather expensive, 30 samples
(a rule of thumb; e.g.~\cite{H19} says ``greater than 25 or 30'') are obtained.
In fact, the 25th percentiles ($Q_1$), 75th percentiles ($Q_3$) and the means of
30, 60, 100, 200 and 1000 samples are not quite different
(Fig.~\ref{Fig:Fay18_10_5_0_0}), although 200 seems a better and relatively
cheaper option.

\begin{figure}
    \centering
        \includegraphics[width=0.88\textwidth]{fig/Fay18_10_5_0_0.pdf}
    \captionsetup{width=.95\textwidth}
    \caption{Testing differences of results from different numbers of samples.
	See Fig.~\ref{Fig:Fay18_10_5bw} for more details.}\label{Fig:Fay18_10_5_0_0}
\end{figure}


\subsection{Extreme Value Study}

It is easy for us to think that less paleomagnetic data means poorer similarity
between paleomagnetic APWPs and the reference path. However, it is noticeable
that even though the data density is tremendously reduced (e.g.\ by 80\%), it is
still possible to have a better similarity for paleomagnetic APWPs and the
reference, even better than for the paleomagnetic APWP with all original
datasets (e.g.\ the black rings in Fig.~\ref{Fig:Fay18_10_5bw}). For example,
for the case (a) (Fig.~\ref{Fig:Fay18_10_5bw}), even though 60\% of the
paleopoles are removed, we still can get a better similarity using the
paleomagnetic APWP composed of the left 40\% of the paleopoles than the
original. Although this 40\% data generated paleomagnetic APWP owns the same
number of mean poles as the 100\% data generated paleomagnetic APWP, the
number of paleopoles for each mean pole is actually much less. The main reason
why this 40\% data generated paleomagnetic APWP is more similar to the
reference path is that this APWP's spatial uncertainties (FQ: C-B) are much
larger than those (FQ: B-B) of the 100\% data generated paleomagnetic APWP
(Fig.~\ref{Fig:Fay18_10_5b101l40p_vs_100p}). Even only 20\% of the paleopoles
could also give a better similarity (the lowest red dot in
Fig.~\ref{Fig:Fay18_10_5bw} (a)). Unfortunately, the reason why this 20\% data
generated paleomagnetic APWP is more similar to the reference path is the same
as for the 40\% data generated path: extremely large spatial uncertainties (FQ:
D-B). So we need to be cautious about the similarity score when we have less
data. It's the same for the lowest difference given by the 60\% Australian
paleomagnetic data (the lowest green dot in Fig.~\ref{Fig:Fay18_10_5bw}
(e)).

\begin{figure*}[tbp]
  \captionsetup[subfigure]{labelformat=empty,aboveskip=-6pt,belowskip=-6pt}
  \centering
  \begin{subfigure}[htbp]{.49\textwidth}
    \captionsetup{skip=0pt}  % local setting for this subfigure
    \centering
    \includegraphics[width=1.01\linewidth]{/home/g/Desktop/git/paper/tex/GeophysJInt/figures/ay18_101comb_10_5_11_0.pdf}
	\caption{100\% data, FQ B-B; Fig.~\ref{Fig:Fay18_10_5bw}
	(a)}\label{Fig:Fay18_10_5b101}
  \end{subfigure}
  \begin{subfigure}[htbp]{.49\textwidth}
    \captionsetup{skip=0pt}
    \centering
    \includegraphics[width=1.01\linewidth]{fig/ay18_101comb_10_5_11_0_40perc_Best.pdf}
    \caption{The lowest for 40\% data, FQ C-B;
	Fig.~\ref{Fig:Fay18_10_5bw} (a)}\label{Fig:Fay18_10_5b101l40p}
  \end{subfigure}
  \caption[Less data, better similarity?]{Comparing the 100\% North American
  120\textendash0 Ma paleomagnetic data derived result with the best of the only
  40\% data (giving even better similarity) derived results (the lowest yellow
  dot in Fig.~\ref{Fig:Fay18_10_5bw} (a)).}\phantomsection\label{Fig:Fay18_10_5b101l40p_vs_100p}
\end{figure*}

For the case (b) (Fig.~\ref{Fig:Fay18_10_5bw}), the main reason why the only
20\% data could give better result than the 100\% data does is that not only,
for example, for the lowest cyan dot the 20\% data gives less mean poles, but
also a few larger spatial uncertainties appear for this 20\% case
(Fig.~\ref{Fig:Fay18_10_5w101l20p_vs_100p}).

\begin{figure*}[tbp]
  \captionsetup[subfigure]{labelformat=empty,aboveskip=-6pt,belowskip=-6pt}
  \centering
  \begin{subfigure}[htbp]{.49\textwidth}
    \captionsetup{skip=0pt}
    \centering
    \includegraphics[width=1.01\linewidth]{/home/g/Desktop/git/paper/tex/GeophysJInt/figures/ay18_101comb_10_5_16_3.pdf}
	\caption{100\% data; Fig.~\ref{Fig:Fay18_10_5bw}
	(b)}\label{Fig:Fay18_10_5w101}
  \end{subfigure}
  \begin{subfigure}[htbp]{.49\textwidth}
    \captionsetup{skip=0pt}
    \centering
    \includegraphics[width=1.01\linewidth]{fig/ay18_101comb_10_5_16_3_20perc_Best.pdf}  %Fay18_101comb_10_5_16_3/_2/026
    \caption{The lowest for 20\% data;
	Fig.~\ref{Fig:Fay18_10_5bw} (b)}\label{Fig:Fay18_10_5w101l20p}
  \end{subfigure}
  \caption[Less data, better similarity?]{Comparing the 100\% North American
  120\textendash0 Ma paleomagnetic data derived result with the best of the only
  20\% data (giving even better similarity) derived results (the green dot in
  Fig.~\ref{Fig:Fay18_10_5bw} (b)).}\phantomsection\label{Fig:Fay18_10_5w101l20p_vs_100p}
\end{figure*}

For the case (c) (Fig.~\ref{Fig:Fay18_10_5bw}), the reason why the 80\% data is
able to give better result than the 100\% data does is that the 10 Ma mean pole
of the 80\% data derived path is a bit closer to reference, because both the 10
Ma pole pair in Fig.~\ref{Fig:Fay18_10_5b501l80p_vs_100p} are distinguishable.
Although the 80\% data derived paleomagnetic APWP
(Fig.~\ref{Fig:Fay18_10_5b501l80p}) generally owns relatively larger spatial
uncertainties, the related pole pairs are distinguishable for both path pairs in
Fig.~\ref{Fig:Fay18_10_5b501l80p_vs_100p}.

\begin{figure*}[tbp]
  \captionsetup[subfigure]{labelformat=empty,aboveskip=-6pt,belowskip=-6pt}
  \centering
  \begin{subfigure}[htbp]{.49\textwidth}
    \captionsetup{skip=0pt}
    \centering
    \includegraphics[width=1.01\linewidth]{/home/g/Desktop/git/paper/tex/GeophysJInt/figures/ay18_501comb_10_5_19_0.pdf}
	\caption{100\% data; Fig.~\ref{Fig:Fay18_10_5bw}
	(c)}\label{Fig:Fay18_10_5b501}
  \end{subfigure}
  \begin{subfigure}[htbp]{.49\textwidth}
    \captionsetup{skip=0pt}
    \centering
    \includegraphics[width=1.01\linewidth]{fig/ay18_501comb_10_5_19_0_80perc_Best.pdf}  %Fay18_501comb_10_5_19_0/_8/005
    \caption{The lowest for 80\% data;
	Fig.~\ref{Fig:Fay18_10_5bw} (c)}\label{Fig:Fay18_10_5b501l80p}
  \end{subfigure}
  \caption[Less data, better similarity?]{Comparing the 100\% Indian
  120\textendash0 Ma paleomagnetic data derived result with the best of the only
  80\% data (giving even better similarity) derived results (the green dot in
  Fig.~\ref{Fig:Fay18_10_5bw} (c)).}\phantomsection\label{Fig:Fay18_10_5b501l80p_vs_100p}
\end{figure*}

Still for the case (c) (Fig.~\ref{Fig:Fay18_10_5bw}), of the 30 samples for the
20\% data, none is able to give better result compared with the 100\% data, but
the lowest difference value we can get from these 30 samples (20\% in
Fig.~\ref{Fig:Fay18_10_5bw} (c)) indicates that 20\% data is still able to give
good (not dramatically different from that 100\% data gives) similarity
(Fig.~\ref{Fig:Fay18_10_5b501l20p}).

\begin{figure*}[tbp]
  \captionsetup[subfigure]{labelformat=empty,aboveskip=-6pt,belowskip=-6pt}
  \centering
  \begin{subfigure}[htbp]{.49\textwidth}
    \captionsetup{skip=0pt}
    \centering
    \includegraphics[width=1.01\linewidth]{/home/g/Desktop/git/paper/tex/GeophysJInt/figures/ay18_501comb_10_5_19_0.pdf}
	\caption{100\% data; Fig.~\ref{Fig:Fay18_10_5bw}
	(c)}\label{Fig:Fay18_10_5b501_}
  \end{subfigure}
  \begin{subfigure}[htbp]{.49\textwidth}
    \captionsetup{skip=0pt}
    \centering
    \includegraphics[width=1.01\linewidth]{fig/ay18_501comb_10_5_19_0_20perc_Best.pdf}  %Fay18_501comb_10_5_19_0/_2/028
    \caption{The lowest for 20\% data;
	Fig.~\ref{Fig:Fay18_10_5bw} (c)}\label{Fig:Fay18_10_5b501l20p}
  \end{subfigure}
  \caption[Less data, better similarity?]{Comparing the 100\% Indian
  120\textendash0 Ma paleomagnetic data derived result with the best of the only
  20\% data (giving even better similarity) derived results (the dark green dot
  in Fig.~\ref{Fig:Fay18_10_5bw}
  (c)).}\phantomsection\label{Fig:Fay18_10_5b501l20p_vs_100p}
\end{figure*}

For the case (d) (Fig.~\ref{Fig:Fay18_10_5bw}), the reason why the only 40\%
data could give better result than the 100\% data does is that for the green
dot (Fig.~\ref{Fig:Fay18_10_5bw} (d)) not only the 40\% data gives less mean
poles (but two ends 120 Ma and 0 Ma still exist), but also the 40\% data does
not contain some ``bad'' paleopoles that are far away from the reference path
(Fig.~\ref{Fig:Fay18_10_5w501l40p}). It's the same for the lowest difference
given by the 20\% data samples (the dark green dot in
Fig.~\ref{Fig:Fay18_10_5bw} (d)).

\begin{figure*}[tbp]
  \captionsetup[subfigure]{labelformat=empty,aboveskip=-6pt,belowskip=-6pt}
  \centering
  \begin{subfigure}[htbp]{.49\textwidth}
    \captionsetup{skip=0pt}
    \centering
    \includegraphics[width=1.01\linewidth]{/home/g/Desktop/git/paper/tex/GeophysJInt/figures/ay18_501comb_10_5_8_3.pdf}
	\caption{100\% data; Fig.~\ref{Fig:Fay18_10_5bw}
	(d)}\label{Fig:Fay18_10_5w501}
  \end{subfigure}
  \begin{subfigure}[htbp]{.49\textwidth}
    \captionsetup{skip=0pt}
    \centering
    \includegraphics[width=1.01\linewidth]{fig/ay18_501comb_10_5_8_3_40perc_Best.pdf}  %Fay18_501comb_10_5_19_0/_2/028
    \caption{The lowest for 40\% data;
	Fig.~\ref{Fig:Fay18_10_5bw} (d)}\label{Fig:Fay18_10_5w501l40p}
  \end{subfigure}
  \caption[Less data, better similarity?]{Comparing the 100\% Indian
  120\textendash0 Ma paleomagnetic data derived result with the best of the only
  40\% data (giving even better similarity) derived results (the green dot in
  Fig.~\ref{Fig:Fay18_10_5bw} (d)).}\phantomsection\label{Fig:Fay18_10_5w501l40p_vs_100p}
\end{figure*}

For the case (e) (Fig.~\ref{Fig:Fay18_10_5bw}), although the 20\% data derived
paleomagnetic paths are not closer to the reference path than the 100\% data
derived one, the closest one (the dark green dot in Fig.~\ref{Fig:Fay18_10_5bw}
(e)) still performs well (Fig.~\ref{Fig:Fay18_10_5b801l20p}). This is mainly
because the number of mean poles becomes less when there are only 20\% of the
paleopoles, especially for 120 Ma and 115 Ma, where the mean poles are far from
the reference path for the 100\% data (Fig.~\ref{Fig:Fay18_10_5b801}), mean
poles are missing for the dark green dot case (Fig.~\ref{Fig:Fay18_10_5bw} (e)).
The same situation happens to the green dot case (Fig.~\ref{Fig:Fay18_10_5bw}
(f)).

\begin{figure*}[tbp]
  \captionsetup[subfigure]{labelformat=empty,aboveskip=-6pt,belowskip=-6pt}
  \centering
  \begin{subfigure}[htbp]{.49\textwidth}
    \captionsetup{skip=0pt}
    \centering
    \includegraphics[width=1.01\linewidth]{/home/g/Desktop/git/paper/tex/GeophysJInt/figures/ay18_801comb_10_5_17_0.pdf}
	\caption{100\% data; Fig.~\ref{Fig:Fay18_10_5bw}
	(e)}\label{Fig:Fay18_10_5b801}
  \end{subfigure}
  \begin{subfigure}[htbp]{.49\textwidth}
    \captionsetup{skip=0pt}
    \centering
    \includegraphics[width=1.01\linewidth]{fig/ay18_801comb_10_5_17_0_20perc_Best.pdf} %Fay18_801comb_10_5_17_0/_2/007/ay18_801comb
    \caption{The lowest for 20\% data;
	Fig.~\ref{Fig:Fay18_10_5bw} (e)}\label{Fig:Fay18_10_5b801l20p}
  \end{subfigure}
  \caption[Less data, better similarity?]{Comparing the 100\% Australian
  120\textendash0 Ma paleomagnetic data derived result with the best of the only
  20\% data derived results (the dark green dot in Fig.~\ref{Fig:Fay18_10_5bw}
  (e)).}\phantomsection\label{Fig:Fay18_10_5b801l20p_vs_100p}
\end{figure*}

For the case (f) (Fig.~\ref{Fig:Fay18_10_5bw}), most of the 20\% data derived
paleomagnetic paths are closer to the reference path than the 100\% data derived
one, especially for the dark green dot case in Fig.~\ref{Fig:Fay18_10_5bw} (f)
(Fig.~\ref{Fig:Fay18_10_5w801l20p}). This is mainly because the number of mean
poles becomes much less when there are only 20\% of the paleopoles, especially
two end mean poles missing.

\begin{figure*}[tbp]
  \captionsetup[subfigure]{labelformat=empty,aboveskip=-6pt,belowskip=-6pt}
  \centering
  \begin{subfigure}[htbp]{.49\textwidth}
    \captionsetup{skip=0pt}
    \centering
    \includegraphics[width=1.01\linewidth]{/home/g/Desktop/git/paper/tex/GeophysJInt/figures/ay18_801comb_10_5_22_3.pdf}
	\caption{100\% data; Fig.~\ref{Fig:Fay18_10_5bw}
	(f)}\label{Fig:Fay18_10_5w801}
  \end{subfigure}
  \begin{subfigure}[htbp]{.49\textwidth}
    \captionsetup{skip=0pt}
    \centering
    \includegraphics[width=1.01\linewidth]{fig/ay18_801comb_10_5_22_3_20perc_Best.pdf} %Fay18_801comb_10_5_22_3/_2/018/ay18_801comb
    \caption{The lowest for 20\% data;
	Fig.~\ref{Fig:Fay18_10_5bw} (f)}\label{Fig:Fay18_10_5w801l20p}
  \end{subfigure}
  \caption[Less data, better similarity?]{Comparing the 100\% Australian
  120\textendash0 Ma paleomagnetic data derived result with the best of the only
  20\% data derived results (the dark green dot in Fig.~\ref{Fig:Fay18_10_5bw}
  (f)).}\phantomsection\label{Fig:Fay18_10_5w801l20p_vs_100p}
\end{figure*}

\section{Are the rules we obtained in the last chapter are still true for less
data?}

First, that if APP is still better, and weighting is still not affecting for
less dense paleomagnetic data, is needed to be tested.

\begin{figure}
    \centering
        \includegraphics[width=1\textwidth]{fig/Fay18_10_5_101appamp.pdf}
    \captionsetup{width=1\textwidth}
    \caption{Comparisons of results from Picking No. 1 (APP) and Picking No. 0
	(AMP) with all the listed weighting methods for three continents. The
	$Q_1$\textendash$Q_3$ interquartile range from Picking No. 1 is also shown
	(shadowed) in the plot of Picking No. 0 for clarity.}\label{Fig:Fay18_10_5_101appamp}
\end{figure}

From the perspective of checking if the $Q_1$\textendash$Q_3$ intervals overlap,
APP is indeed still better than AMP and weighting is indeed still not affecting
for more than 15 paleopoles making a 10/5 Myr bin/step APWP (ideally composed of
25 mean poles for 120\textendash0 Ma). For the 20\% Indian data case, which
contains not more than 15 paleopoles making a 120\textendash0 Ma APWP, there is
overlapping between APP's Mean\textendash$Q_3$ interval and AMP's
$Q_1$\textendash{}Mean interval for weighted cases (i.e.\ for Weighting No.
1\textendash5). Even so, APP's means are still lower than AMP's for this
no-more-than-15-paleopoles case (Fig.~\ref{Fig:Fay18_10_5_101appamp}). So here
is the question: is 15 the threshold?

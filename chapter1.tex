\chapter{Introduction}\label{chap:Intro}
\textit{The first chapter introduces paleomagnetism-based paleogeographic
reconstruction technique and highlights the motivation of the research conducted
in the dissertation}
\vfill
\minitoc\newpage

\section{Background and Motivation}

Reconstructing past paleogeographies, especially the motion of plates and their
interactions through time, is a key component of understanding the Earth's
geological history, including deciphering tectonics (e.g.\ supercontinent
reconstruction), paleo-climate history, and the evolution of life. Since the
advent of plate tectonics, it has been the background for nearly all geologic
events. In addition, plate reconstructions form the basis of global or regional
geodynamic models.

\subsection{Techniques Used in Relative and Absolute Plate Motion Studies}

The earliest quantitative effort to model plate kinematics was fitting conjugate
passive margins of the Atlantic~\citep{B65,W07}. They showed that the Atlantic
could be closed using a single Euler pole (using Euler's theorem on rotation).
Then it became fitting conjugate isochrons based on best-fitting marine magnetic
anomaly and fracture zone data~\citep{M71}, which minimizes the misfit area
between two isochrons. The \emph{Hellinger} method~\citep{H81} is a more advanced
and generalised method which also fits conjugate isochrons based on best-fitting
marine magnetic anomaly and fracture zone data, which however minimizes the sum
of the misfits of conjugate data points that belong to a common isochron
segment~\citep{W07} instead. These techniques mainly through fitting conjugate
lines mentioned above are relatively accurate for quantitative analysis. However
they give relative, not absolute, motions between plates, because plate motions
can't be tied into absolute location on Earth's surface, since both plates are
likely moving. In addition, they are limited to survey data from the seafloor,
with a maximum age of no more than c. 200 Ma~\citep{M08}.

Reference frames are a means of describing the motion of geologic features
(e.g.\ tectonic plates) on the surface of the Earth, relative to a common point
or ``frame''~\citep{Sh12}. An absolute reference frame is a frame that can be
treated as fixed relative to the Earth's geographic reference frame. In reality,
it's impossible to find a truly absolute reference frame, so we are actually
looking for a frame that has limited (and hopefully known) motion, which
approximates as ``fixed'' over geologically useful timescales and provides the
most complete descriptions of plate motions. A commonly used absolute reference
frame is the ``Fixed Hotspot model''~\citep{M93,M99}, covering ages from about
132 Ma to present-day, which assumes that the linear volcanic chains found on
most oceanic plates are artifacts of absolute plate motions over a upwelling
plume from the deep mantle, which is assumed to be relatively fixed. The
advantage of this ``Fixed hotspot model'' is that it is fairly straightforward
if the assumption of fixed hotspots is correct. However, this model is limited
to plates with well-dated volcanic hotspot chains~\citep[e.g.\ the Ninetyeast
Ridge on the Indian Ocean floor and the Walvis Ridge in the southern Atlantic
Ocean; see][]{O05} and dating can be difficult~\citep[e.g.\ diffuse volcanic
centers possibly related to large diameter plume conduits could cause the
existence of time reversals; see][]{O05}. As for not well-dated hotspot tracks,
for example, only about 5\textperthousand\ of the seamounts (thought to be
volcanic) in the Pacific are thought be be related to hotspot volcanism and
radiometrically dated~\citep[39 per cent of these ages are less than 10 Ma;
see][]{H07}. In addition, the fixed hotspot model is mostly confined to existing
oceanic or thin continental crust because older oceanic lithosphere has been
largely destroyed by subduction and old, thick continental crust mostly removed
by erosion~\citep{C13}. Last, but not least, hotspots can be susceptible to
drift that may be caused by changes in sub-lithospheric mantle flow~\citep{T09}.
Generally, however, the drift rate is considered to be an order of magnitude
less than the rate of plate motions, so only becomes significant over timescales
of c. 50 Myr or more~\citep{O05,T07}. To overcome this source of error, the
``Moving hotspot model''~\citep{O05} uses mantle convection modeling to predict
hotspot drift. Some are apparent success, e.g.\ by getting motions in the
Indo-Atlantic and Pacific hotspot clusters to agree with each other, but it's
very dependent on the mantle model. Hybrid models attempt to overcome the
shortcomings of each reference frame by combining them, e.g.\ combining a fixed
hotspot frame from 100 Ma to 0 Ma~\citep{M93} with a moving hotspot frame from
c. 132\textendash100 Ma~\citep{O05} (Hybrid hotspot model~\citep{Sh12}),
combining a moving hotspot frame from 100\textendash0 Ma~\citep{O05} with a
paleomagnetic model (reflect plate motion relative to the magnetic dipole axis
but cannot provide paleolongitudes because of the axial symmetry of the Earth's
magnetic dipole field)~\citep{T08} from 140\textendash100 Ma (Hybrid
paleomagnetic model~\citep{Sh12}), and combining a moving hotspot frame from
120\textendash100 Ma~\citep{O05} with a True Polar Wander (TPW) corrected
paleomagnetic model~\citep{S08} from 100\textendash0 Ma (Hybrid TPW-corrected
model~\citep{Sh12}).

Recently another absolute reference frame ``Subduction reference
model''~\citep{v10} tries to connect orogenies/sutures/subduction complexes' on
the Earth surface with their corresponding subducted slabs in the mantle.
Assuming that these remnants sank vertically through the mantle, the absolute
location at which they were subducted can be reconstructed. In this way, this
model mainly imposes a longitude correction on the above mentioned ``Hybrid
TPW-corrected model'', and can theoretically give past absolute locations of
plates back to about 260 Ma based on the estimated age of the oldest slab
remnants that can be reliably located in the mantle. While the ``Subduction
reference model'' allows for reconstructions between about 260 Ma and 140 Ma,
older than the other absolute models can predict, the model is strongly
dependent on the vertical subduction assumption and resolution of seismic
tomography models, so its uncertainty is high. Above all, importantly, if we can
describe the absolute motion of one or a few key plates, the techniques for
establishing the relative plate motions described in the second paragraph above
can be used to construct plate circuits that allow a full kinematic description
of plate tectonics to be developed.

As we can see, all of these above reconstruction methods are limited to recent
geological history. For most of Earth history, concretely for times before c.
170 Ma, the age of the oldest magnetic anomaly identification, paleomagnetism
is the only accepted quantitative method for reconstructing plate motions and
past paleogeographies.

\subsection{Application of Paleomagnetism to Plate Tectonics}

\begin{figure}
  \centering
    \includegraphics[width=0.88\textwidth]{../sphinx/source/I/fig/GAD.pdf}
  \captionsetup{width=.95\textwidth}
  \caption[Geocentric axial dipole (GAD) model]{GAD model: Inclination (angle I
  = $tan^{-1}(2tan\lambda)$) of the Earth's magnetic field and how it varies
  with latitude, redrawn from~\citet{B92,T19,T08}. Magnetic dipole M is placed
  at the center of the Earth and aligned with the rotation axis; $\lambda$ is
  the geographic latitude, and $\theta$ is the
  colatitude.}\label{Fig:chap_intro_gad}
\end{figure}

The geomagnetic field is generated by the convective flow of a liquid
iron-nickel alloy in the outer core of the Earth. It is largely dipolar and can
be represented by a dipole that points from the north magnetic pole to the south
pole. However, the geomagnetic field varies in strength and direction over
decadal\textendash{}millennial timescales due to quadropole and octopole
components of the field. The most spectacular variations in direction are
occasional polarity reversals (normal polarity: the same as the present
direction of the field; or the opposite, i.e.\ reverse polarity). Over a period
of a few thousand years, the magnetic axis slowly rotates/precesses around the
geographic axis and the Earth's rotation axis (secular variation), but when
averaged over 10,000 year timescales, higher order components of the field are
thought to largely cancel out and the position of the magnetic poles aligns with
the geographic poles. This is the geocentric axial dipole (GAD) hypothesis. In a
GAD field, at the north magnetic pole the inclination (angle with respect to the
local horizontal plane, see Fig.~\ref{Fig:chap_intro_gad}) of the field is
+90\degree\ (straight down), at the Equator the field inclination is 0\degree\
(horizontal) pointing north and at the south magnetic pole the inclination is
-90\degree\ (straight up) (Fig.~\ref{Fig:chap_intro_gad}). Another direction
parameter of the Earth's magnetic field is declination. It is the angle with
respect to the geographic meridian, which is 0\degree\ everywhere in a
time-averaged GAD field.

Magnetic remanence is the magnetization left behind in a ferromagnetic substance
in the absence of an external magnetic field~\citep{T19}. The remanent
magnetisation of rocks can preserve the direction and intensity of the
geomagnetic field when the rock was formed, e.g.\ in the process of cooling,
ferromagnetic materials in the lava flow are magnetized in the direction of the
Earth's magnetic field, so the local direction of the field vector is locked in
solidified lava. We are often interested in whether the geomagnetic pole has
changed, or whether a particular plate/terrane has rotated with respect to the
geomagnetic pole~\citep{T19}. By measuring the direction of the remanent
magnetisation, we can calculate a virtual geomagnetic pole (VGP) to represent
the geomagnetic pole of an imaginary geocentric dipole which would give rise to
the observed remanent declination and inclination. Collection of VGPs (or
site-mean directions) allow calculating a ``paleomagnetic pole'', also known as
paleopole, at the formation level. Commonly a paleopole is a Fisherian
mean~\citep{F53} with a spatial uncertainty. A paleopole that plots away from the
present geographic poles is assumed to be due to plate motions since the lava
was solidified, which causes the paleopole to move with the plate~\citep{T08}.
Based on measurements of the remanent inclination, the ancient latitude for a
plate can be calculated when the rock formed from the dipole formula $\tan(I) =
2 *\tan(latitude)$. In addition, the remanent declination provides information
about the rotation of a plate. Ideally, as a time average, a paleopole (which
can be calculated from declination, inclination and the current geographic
location of the sampling site) for a newly formed rock will correspond with the
geographic north or south pole. To perform a reconstruction with paleopoles we
therefore have to calculate the rotation (Euler) pole and angle which will bring
the paleopole back to the geographic north or south pole, and then rotate the
plate by the same amount of angle using the same Euler pole. This is how
paleomagnetism can be used to reconstruct past positions of a plate. In our
example (Fig.~\ref{Fig:chap_intro_reconstructpole}), a c.155 Ma paleopole
(latitude=52.59\degree{}N, longitude=91.45\degree{}W) will be restored to the
geographic pole by an Euler rotation of pole (0\degree, 178.55\degree{}E) with
angle 37.41\degree, which rotates the sampling site from its present position of
(0\degree, 25\degree{}E) to the Africa paleo-continent at (15.6998\degree{}S,
20.1121\degree{}E). So Africa must have drifted northwards since the Late
Jurassic.

\begin{figure}
  \centering
    \includegraphics[width=0.88\textwidth]{../sphinx/source/I/fig/af.pdf}
  \captionsetup{width=.95\textwidth}
  \caption[The hemispheric ambiguity and absolute paleolongitude indeterminacy
  with a single paleomagnetic pole (paleopole)]{Reconstruction of Africa with
  its c. 155 Ma paleopole. The red polygon is today's position of Africa, while
  the blue and green ones shows its reconstructed position at c. 155 Ma, if the
  pole was North and South pole, respectively. Dashed green polygon illustrates
  the ambiguity of paleolongitude from paleomagnetic data alone (sites at same
  latitude but different longitudes record the same Declination and Inclination
  in a GAD field).}\label{Fig:chap_intro_reconstructpole}
\end{figure}

However, there are 2 problems with using paleomagnetic poles for constraining
finite rotations~\citep{T19}. First, if only one paleomagnetic pole is
given alone without any geologic context, its polarity can be ambiguous, i.e.\
an upward inclination may be due to being located in the southern hemisphere
during a normal polarity chron, or in the northern hemisphere during a reversed
polarity chron (cf.\ the solid blue and solid green Africa in Fig.~\ref{Fig:chap_intro_reconstructpole}). In other words, we can't know if it's North pole
or South pole, especially for paleomagnetic data with the Precambrian and early
Paleozoic ages. Returning to the example above, if the c. 155 Ma paleomagnetic
pole (52.59\degree{}N, 91.45\degree{}W) was formed during a period of reversed polarity, then
it needs to be rotated to the South pole rather than the North pole. The
necessary Euler rotation of pole (0\degree, 1.45\degree{}W) and angle
142.59\degree\ rotates the sampling site (0\degree, 25\degree{}E) on Africa to
(15.6998\degree{}N, 23.0121\degree{}W) indicating
southward motion since the Late Jurassic. Second, because in a GAD field the
declination equals zero everywhere (Fig.~\ref{Fig:chap_intro_reconstructpole}),
paleomagnetic data doesn't register longitudinal motions of plates (the Euler
pole for a plate moving purely to the east or west is at the geographic poles,
so preserved paleomagnetic poles will experience zero rotation), which means we
can position a plate at any longitude we wish subject to other geological
constraints (cf.\ the solid and dashed green Africa in Fig.~\ref{Fig:chap_intro_reconstructpole}).

The data source used in this dissertation is \emph{Global Paleomagnetic
Database} (GPMDB) Version 4.6b~\cite[updated in 2016 by the Ivar Giaever
Geomagnetic Laboratory team, in collaboration with Pisarevsky]{M96,P05}, which
includes 9514 paleopoles for ages of 3,500 Ma to the present published from 1925
to 2016. GPMDB has been published in two ways: (1) IAGA GPMDB 4.6 online query:
\emph{http://www.ngu.no/geodynamics/gpmdb/}, which is now closed; (2) Microsoft
Access system in \emph{.mdb} format at NOAA's National Geophysical Data Center
\emph{https://www.ngdc.noaa.gov/geomag/paleo.shtml}~\citep{P03}
and CESRE's Paleomagnetism and Rock Magnetism project
\emph{https://wiki.csiro.au/display/cmfr/Palaeomagnetism+and+Rock+Magnetism},
which is later updated by Ivar Giaever Geomagnetic Laboratory
\emph{http://www.iggl.no/resources.html\#data}.

\begin{figure}
\centering
\includegraphics[width=1\textwidth]{/home/i/Desktop/git/pps/poster/fig/na.pdf}
\captionsetup{width=.95\textwidth}
\caption{Much paleomagnetic data has been collected from the North American
Craton. For younger geologic times, do we really need so much data to
reconstruct accurately just like modern-day plate motions? The image shows
distribution of all published paleomagnetic poles of the NAC over time, which
are compiled from GPMDB 4.6b~\citep{P05} and
PALEOMAGIA~\citep{V14}.}\label{Fig:chap_intro_nacpole}
\end{figure}

An apparent polar wander path (APWP) is composed of poles of different ages
from different sampling sites on the same stable (non-deforming) continent,
chained together to form a record of motion relative to the fixed magnetic pole
over geological time. It represents a convenient way of summarizing
paleomagnetic data for a plate instead of producing paleogeographic maps at
each geological period~\citep{T08}. As a preliminary study, the \emph{North
American Craton} (NAC) is chosen as a research object to develop techniques we
want to think about. The NAC is one of best studied cratons in paleomagnetism
with the GPMDB containing 2160 poles published since 1948
(Fig.~\ref{Fig:chap_intro_nacpole}). If we observe the latitudes, longitudes and
age distribution of the NAC poles (Fig.~\ref{Fig:chap_intro_nacpole}), we
actually can identify the general trend of its APWP\@. However, converting this
data into a reliable, well-defined APWP can be challenging, due to the following
issues:

\subsection{Fact 1: Not All Regions on the Earth Surface Are Solid}

If we consider the modern North America continent, the region west of the
Rockies is actively deforming. Paleomagnetic data from such areas are likely to
reflect local tectonic processes such as block rotation rather than rigid plate
motions, and should be excluded. For example, the Rockies Mountain area was not
included as my data selecting polygon (the transparent yellow area in
Fig.~\ref{Fig:chap_intro_nacpole}). In order to investigate a specific craton or
terrane or block's past paleogeographic motion, choosing an appropriate
subregion without active tectonic activities, e.g.\ rotation or uplifting or
rifting, to select data is often required. Such tectonics-free regions are
usually called rigid. However, the difficulty of defining such tectonic
boundaries makes appropriate spatial and temporal choices very difficult,
particularly further in the geological past when cratonic configurations and
active plate boundaries were very different to today. This leads to a question:
What is the best way to constrain the data for a specific plate or block? The
solution proposed in this dissertation is described in Appendix A of Chapter 3.

\subsection{Fact 2: Not All Data Are Created Equal}

APWPs are generated by combining paleomagnetic poles for a particular rigid
block over the desired age range to produce a smoothed path. However, the NAC
dataset illustrates that uncertainties in the age and location of paleomagnetic
poles in the GPMDB can vary greatly for different poles.

\subsubsection{Age Error}

Although remanent magnetizations are generally assumed to be primary, many
events can cause remagnetisation (in which case the derived pole is `younger'
than the rock). If an event that has occurred since the rock's formation that
should affect the magnetisation (e.g.\ folding, thermal overprinting due to
intrusion) can be shown to have affected it, then it constrains the
magnetisation to have been acquired before that event. Recognising or ruling
out remagnetisations depends on these field tests, which are not always
performed or possible. Even a passed field test may not be useful if field test
shows magnetisation acquired prior to a folding event tens of millions of years
after initial rock formation.

The most obvious characteristic we can observe from NAC paleomagnetic data
(Fig.~\ref{Fig:chap_intro_nacpole}) is that some poles have very large age
ranges, e.g.\ more than 100 Myr. The magnetization age should be some time
between the information of the rock and folding events. There are also others
where we have similar position but the age constraint is much narrower, e.g.\ 10
Myr window or less. Obviously the latter kind of data is more valuable than the
one with large age range.

\subsubsection{Position Error}

The errors of pole latitudes and longitudes are plotted as 95\% confidence
ellipses (Fig.~\ref{Fig:chap_intro_nacpole}), which also vary greatly in magnitude.
All paleomagnetic poles have some associated uncertainties due to measurement
error and the nature of the geomagnetic field. More uncertainties can be added
by too few samples, sampling spanning too short a time range to approximate a
GAD field, failure to remove overprints during demagnetisation, etc.

\subsubsection{Data Consistency}

Paleomagnetic poles of a rigid plate or block should be continuous time series.
For a rigid plate, two poles with similar ages shouldn't be dramatically
different in location. We want to look at the consistency of NAC and India's
data over smaller time periods, so the data is binned over a small time interval
(e.g.\ 2 Myr) to see whether the paleopoles in each time interval overlap within
their error ellipses, as they should. Sometimes, this is the case
(Fig.~\ref{Fig:chap_intro_na6462agemean}). Sometimes we have further separated
poles with close ages (Fig.~\ref{Fig:chap_intro_in97agemean}).

\begin{figure*}[tbp]
  \captionsetup[subfigure]{labelformat=empty,aboveskip=-6pt,belowskip=-6pt}
  \centering
  \begin{subfigure}[htbp]{.49\textwidth}
    \captionsetup{skip=0pt}  % local setting for this subfigure
    \centering
    \includegraphics[width=1.01\linewidth]{../sphinx/source/I/fig/na6462.pdf}
    \caption{North America: 64\textendash62 Ma (mean age)}\label{Fig:chap_intro_na6462agemean}
  \end{subfigure}
  \begin{subfigure}[htbp]{.49\textwidth}
    \captionsetup{skip=0pt}
    \centering
    \includegraphics[width=1.01\linewidth]{../sphinx/source/I/fig/in97.pdf}
    \caption{India: 9\textendash7 Ma (mean age)}\label{Fig:chap_intro_in97agemean}
  \end{subfigure}
  \caption[Example of AMP Moving Averaging Effects]{Overlapping and further
  separated paleomagnetic poles of NAC\@ and India. The oval ellipses are their
  95\% confidence errors. The labels are their result number given in GPMDB
  4.6b.}\phantomsection\label{Fig:chap_intro_ma-amp}
\end{figure*}

There are a number of possible causes for these outliers, including:

\paragraph{Lithology}

For this poor consistency of data (Fig.~\ref{Fig:chap_intro_in97agemean}), it
is potentially because of different inclinations or declinations. The first
thing we should consider about is their lithology. We want to check if the
sample rock are igneous or sedimentary, because sediment compaction can result
in anomalously shallow inclinations~\citep{T19}. In addition, we also
can check if the rock are redbeds or non-redbeds. Although whether redbeds
record a detrital signal or a later chemical remanent magnetization (CRM) is
still somewhat controversial, both sedimentary rocks and redbeds could lead to
inconsistency in direction compared to igneous rocks. For this case, all the
three poles (Fig.~\ref{Fig:chap_intro_in97agemean}) are from sedimentary
rocks. In addition, pole 1136 and 1137 (Result Number in GPMDB 4.6b)'s source
rocks also contain redbeds~\citep{O82}, although the authors did not mention
about the potential inclination shallowing. For pole 7095, although the source
rocks do not contain redbeds, the authors did mention about possible inclination
shallowing due to haematite grains~\citep{G94}.

\paragraph{Local Rotations}

As discussed previously, local deformation between two paleomagnetic localities
invalidates the rigid plate assumption and could lead to inconsistent paleopole
directions. All the three poles (Fig.~\ref{Fig:chap_intro_in97agemean})
contain signals of local rotations~\citep{O82,G94}, e.g.\ pole 7095 has a signal
which suggests the presence of a counter-clockwise local rotation of the Tinau
Khola section~\citep{G94}, and therefore do not reflect motions of the whole
rigid India plate in this case. So the discordance is likely due to local
deformation (Fig.~\ref{Fig:chap_intro_in97agemean}), and we would ideally
want to exclude or correct such poles from our APWP calculation.

\paragraph{Other Factors}

In Fig.~\ref{Fig:chap_intro_ma-amp}, mean pole age (centre of age error) has
just been binned. If any of the paleopoles have large age errors, they could be
different ages from each other and sample entirely different parts of the
APWP\@. Conversely, if any of the paleopoles have too few samples, or were not
sampled over enough time to average to a GAD field, a discordant pole may be due
to unreduced secular variation, because in order to average errors in
orientation of the samples and scatter caused by secular variation, a
``sufficient'' number of individually oriented samples from ``enough'' sites
must be satisfied~\citep{T19,v90,B02}. For example, pole 1136
(Fig.~\ref{Fig:chap_intro_in97agemean}) is from only 4 sampling sites, pole 1137
from only 3 sites and pole 7095's site number not even given in the GPMDB 4.6b.

\subsubsection{Data Density}

As we go back in time, we have lower quality and lower density (or quantity) of
data, for example, the Precambrian or Early Paleozoic paleomagnetic data are
relatively fewer than Middle-Late Phanerozoic ones, and most of them are not
high-quality, e.g.\ larger errors in both age and location
(Fig.~\ref{Fig:chap_intro_nacpole}). The combination of lower data quality with
lower data density means that a single `bad' pole (with large errors in age
and/or location) can much more easily distort the reconstructed APWP, because
there are few or no `good' poles to counteract its influence.

Data density also varies between different plates. E.g., we have a relatively
high density of paleomagnetic data for NAC, but few poles exist for Greenland
and Arabia. Based on mean age (mean of lower and upper magnetic ages), for
120\textendash0 Ma, GPMDB 4.6b has more than 130 poles for NAC, but only 17 for
Greenland and 24 for Arabia.

\subsubsection{Publication Year}

The time when the data was published should also be considered, because
magnetism measuring methodology, technology and equipments have been improved
since the early 20th century. For example, stepwise demagnetisation, which is
the most reliable method of detecting and removing secondary overprints, has
only been in common use since the mid 1980s.

In summary, not all paleopoles are created equal, which leads to an
important question: how to best combine poles of varying quality into a
coherent and accurate APWP\@? Paleomagnetists have proposed a variety of methods
to filter so-called ``bad'' data, or give lower weights to those ``bad'' data
before generating an APWP, e.g.\ two widely used methods: the V90 reliability
criteria~\citep{v90} and the BC02 selection criteria~\citep{B02}. Briefly, the V90
criteria for paleomagnetic results includes seven criteria: (1) Well determined
age; (2) At least 25 samples with Fisher~\citep{F53} precision $\kappa$ greater
than 10 and $\alpha$95 less than 16\degree; (3) Detailed demagnetisation results
reported; (4) Passed field tests; (5) Tectonic coherence with continent and good
structural control; (6) Identified antipodal reversals; (7) Lack of similarity
with younger poles~\citep{T92}. Compared with V90, the BC02 criteria suggests
stricter filtering, e.g.\ using only poles with at least 6 sampling sites and 36
samples, each site having $\alpha$95 less than 10\degree\ in the Cenozoic and
15\degree\ in the Mesozoic. There are many potential ways to weight the data set
which could obviously greatly influence the final result, and we want to test
this. But there has been limited study of how effective these
filtering/weighting methods are at reconstructing a `true' APWP, and for most
studies after a basic filtering of `low quality' poles, the remaining poles are,
in fact, treated equally.

\section{Objectives}

Our overarching aims are to develop rigorous, consistent and well-documented
methods of reconstructing plate motions using paleomagnetic data, and to
investigate the limits of paleomagnetic data on reconstructing individual plate
motions, supercontinents, and global tectonic parameters like average rate of
plate motion.

\subsection{Motivation and General Approach}

How has plate tectonics evolved over geologic history, in terms of average plate
velocities, numbers of plates and so on? The only quantitative data we have
prior to about 170 Ma are paleomagnetic data. We know there are limitations,
because we can't constrain the longitudes of paleo-plates very well. When we
look back through geologic history, how much good paleomagnetic data do we have,
and how well does it reconstruct `true' plate motions? We don't know well the
effects of data quality and density, which generally degrades further back in
geologic history, on producing reliable APWPs. For the past c. 130\textendash200
Myr we have the highest density of paleomagnetic data and also independent plate
motion data from reconstructions of ocean spreading combined with hotspot
reference frames. These independent data sources can help constrain plate
motions in more accurate ways. This allows us to ask the question: How much
paleomagnetic data do we need actually to reconstruct accurately known
modern-day plate motions? If we can handle that, we can go back in time. For a
certain density of paleomagnetic data that we have, how reliably can we talk
about what's going on in the past given the much lower data distribution? It
might turn out we don't need very much data to say something reasonably and
reliably. We can test this by looking at the last 0\textendash120 Ma where we
can compare paleomagnetically derived plate motions with other methods of
paleogeographic reconstruction. This does not only include the work of
developing tools and algorithms to generate those paleomagnetically derived
plate motions (to use paleomagnetic data to reconstruct APWP parameters that are
known from other sources like ocean basins and hotspots), but also need us to
know how good these tools are or which one is the best algorithm (to compare
paleomagnetic APWPs with the known data sources predicted APWP). This can give
insights into how well we can `know' plate motions back in the past, and what
data quality and density are necessary to reliably reconstruct a `true' APWP\@.

As a preliminary analysis, some algorithms were made to separate/calculate out
so-called good paleomagnetic data (at any particular time period for a
particular craton, like here from 100 Ma to the present day for NAC). We are
interested in what makes `good' data, how we can identify it and filter it from
the database, and how sometimes `bad' data is only bad in the sense that it is
poorly constrained in age or position or any other parameter, in which cases it
might be possible to include it by e.g., weighting. A weighted mean pole can be
calculated for a time interval with `better' (more likely to be reliable) poles
counting more than `worse'. For example, a pole with small $\alpha$95 and very
well constrained age is more likely to reflect APWP position at the selected age
point than a pole with large $\alpha$95 and very broad age range.

\subsection{Research Questions or Hypotheses}

Questions 1\textendash4 focus on method development, whereas 5 and 6 start
using them for plate tectonic research, especially in deep times.

\subsubsection{Question 1}

What is the best way to turn a collection of individual poles, with different
age constraints and uncertainties, into a smoothed APW path? This question, in
fact, is about how to (1) choose a data-constraining polygon that represents a
solid continent during a certain period; (2) pick (or bin) data within a certain
window for Fisher statistical~\citep{F53} calculation; (3) do weighting for
picked data according to different uncertainties or other kinds of standards of
qualifications; (4) if the derived APWP is still not smoothed enough when
compared with a reference path, is further smoothing necessary? Our goal here
actually is to get a reliable result, i.e.\ a path generated to approximate the
`real' APWP with appropriate uncertainties.

\subsubsection{Question 2}

Based on the consequences from the algorithms we developed, we can do research
on why some algorithms are good, others bad for all plates? Why some algorithm
performs well for a plate or two but not others?

\subsubsection{Question 3}

How much paleomagnetic data do we need actually to accurately reconstruct known
modern known plate motions? What insights does this give us into the
reliability of reconstructions from earlier in geologic history?

\subsubsection{Question 4}

Based on our analysis above, can we develop algorithms that look for matching
segments of APWPs from different cratons, that might indicate they were part of
the same continent or supercontinent?

\subsubsection{Question 5}

What kind of dataset (in terms of data density and quality) is needed to
accurately reconstruct a known APWP, or a shared APWP between two cratons? If
we can establish some criteria for this, does it provide any insights into past
reconstructions of plate motions (e.g., Rodinia)?

\subsubsection{Question 6}

Can we develop algorithms that use APWPs from multiple continents to estimate
global average plate motion rates? Can we get a good sense of how much
information is lost due to lack of data on longitudinal motions? Can we use
this to draw any conclusions about long term trends (or lack thereof) in the
style and vigour of global plate tectonics? (Possible further question: can
data on relative continental motion acquired from matching APWP curves be
incorporated to improve these estimates?)

In summary, this dissertation will not be able to help answer all the above
questions. However, in the end the completion of this dissertation and solving
the first three questions are hoped to be helpful solving the later questions in
the future.
